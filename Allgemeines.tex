\section{Allgemeines} 

\subsection{Bases (S. 3)}

\begin{itemize}

 \item Grundsätzlich werden runde, ovale oder abgerundete Bases verwendet. Als
  Anhaltspunkt für die Basegröße dient die in der aktuellen Verpackung
  enthaltene Base.

\item Bestien stehen in der Regel auf einem 40mm-Rundbase. Bikes und Kavallerie
 auf den langen, abgerundeten Bikebases.

\end{itemize}

\subsection{Umbauten}

\begin{itemize}

 \item Umbauten/Selbstbauten sind grundsätzlich zugelassen. Sie müssen in Größe
  und Erscheinung dem Original ähneln, damit erkennbar ist, um was es sich
  handelt.
 Beispiele:
\begin{itemize}
 \item Kairos hat 2 Köpfe
 \item Ein SM-Scriptor hat eine Psimatrix
 \item Vulkan He'stan hat einen Speer
\end{itemize}

\item Einzelne Umbauten können von der Orga abgelehnt werden. Grundsätzlich
 stehen Umbauten  auf entsprechenden Bases.

\end{itemize}

\subsection{Zeitspiel}

\begin{itemize}

 \item Zeitspiel kann bestraft werden. Der Schiedsrichter entscheidet nach
  eigenem Ermessen, ob er Zeitspiel während oder nach dem Spiel ahndet.

\item Dies kann auf mehrere Arten erfolgen:
\begin{itemize}
 \item Eine Verwarnung durch den Schiedsrichter
 \item Der Schiedsrichter beendet den aktuellen Zug des Spielers
 \item Wenn ein Spiel über nur 3 Runden ging, kann der Schiedsrichter bis zu 5
  Strafpunkte verteilen
 \item Wenn ein Spiel über nur 4 Runden ging, kann der Schiedsrichter bis zu 3
  Strafpunkte
\end{itemize}

\item Wenn die Spielzeit abgelaufen ist, gibt es eine 3-minütige Deadline.
 Sollte der Ergebniszettel beim Erlöschen der Deadline der Orga nicht vorliegen,
 so wird das Spiel als 0:0 gewertet.

\end{itemize}


