\section{Codizes}

\subsection{Chaos Dämonen}

\begin{itemize}

 \item Wenn auf der Warpsturmtabelle ein Ergebnis ermittelt wurde, bei dem ein
  W6 für jede nicht im Nahkampf gebundene gegnerische Einheit (sowie für Dämonen
  bestimmter Gottheiten) gewürfelt werden muss, wird nicht für Einheiten in
  Transportfahrzeugen oder in Gebäuden gewürfelt.

 \item Das Ergebnis ``Dämonische Besessenheit'' von der Warpsturmtabelle hat
  keinen Effekt auf Psioniker, welche sich in Transportfahrzeugen oder in
  Gebäuden befinden. Fliegende monströse Kreaturen im Jagdflug (die gleichzeitig
  Psioniker sind) werden normal betroffen.

 \item Wenn eine neue Einheit Dämonen durch die Portalglyphe oder die
  Warpsturmtabelle generiert wird, darf der Dämonen Spieler die Anzahl der
  Dämonen auswürfeln, bevor er sich für die Art der Dämonen entscheidet. 

 \item Wenn während des Spiels ein neuer Psioniker (oder eine Einheit Psioniker)
  generiert wird, würfle sofort die zufällig bestimmten Psi-Kräfte aus, bevor du
  die Modelle auf dem Spielfeld platzierst. Psioniker, die auf diese Weise das
  Spielfeld betreten, besitzen Warpenergie entsprechend ihrem
  Meisterschaftsgrad.

 \item Unabhängige Charaktermodelle aus dem Codex: Chaosdämonen dürfen sich
  nicht Einheiten aus dem Codex: Chaos Space Marines anschließen.

 \item Verletzungen ignorieren kann gegen Wunden, die durch dämonische
  Instabilität hervorgerufen, genutzt werden.

 \item Kairos Schicksalswebers Wiederholungswurf (den er durch den Stab des
  Morgen erhält), kann nicht genutzt werden, um den Wurf um die 6. Und 7. Runde
  zu wiederholen. Ebenso kann der Wurf zur Ermittlung des Nachtkampfes sowie
  Würfe des Gegners nicht mithilfe des Stabes neu gewürfelt werden.

 \item Kairos Schicksalsweber wird bezüglich Sonderregeln oder Angriffen, die
  Psioniker betreffen, immer als einzelner Psioniker behandelt.
 \item Kairos Schicksalswebers Wiederholungswurf, der ihm durch den Stab des
  Morgen gewährt wird und die Kriegsherrenfähigkeit Fürst der Unrealität können
  auch genutzt werden, wenn sich Kairos in Reserve befindet. 
 
 \item Wenn Kreischer ihre normalen Nahkampfattacken gegen eine einzelne
  Spezial-Attacke mit Warpmaul ersetzen, dann verfällt in diesem Falle auch die
  Bonusattacke, für den Nahkampfangriff. Niederschmettern ist hiervon nicht
  betroffen.

 \item Deckungswürfe gegen Tiefflugangriffe der Kreischer sind nur möglich, wenn
  sich die betroffenen Modelle in Geländezonen befinden (Fahrzeuge
  ausgeschlossen) oder wenn eine Sonderregel dies erlaubt.

 \item Das Blaue und Rosa Feuer des Tzeentch sind Dämonengeschenke, die zum
  erhabenen Feuerdämon gehören, nicht zum brennenden Streitwagen. Das bedeutet,
  wenn sich der Streitwagen bewegt, kann in der Schussphase das blaue Feuer nur
  als Schnellschuss und das Rosa Feuer nicht abgefeuert werden.

 \item Wenn die Maske des Slaanesh den Tanz des Fesselns gegen eine Einheit
  einsetzt, wird für jede Bewegung, die die Einheit macht, einzeln gewürfelt.
  Das bedeutet, bewegt sich die Einheit, würfle einen W3. Rennt die Einheit,
  würfle einen W3 usw. Es wird bei jeder Bewegung neu gewürfelt und der Tanz
  wirkt sich auf jede Art von Bewegung aus, auch Turbo-Boost und Vollgas.
  Fliegende monströse Kreaturen, die durch den Tanz des Fesselns betroffen sind,
  können sich nicht im Jagdflug bewegen.

 \item Der verstörende Gesang einer Slaanesh Bestie verursacht -1 auf den
  Moralwert, für jede Bestie die sich in 12 Zoll Umkreis um einen Psioniker
  befindet. Dieser Modifikator ist kumulativ mit anderen Bestien/Einheiten, die
  über die Sonderregel verstörender Gesang verfügen. Beachte, das ein Moralwert
  Test nicht das Gleiche ist wie ein Profilwerttest, daher ist es mit einer
  Doppel 1 immer möglich, einen Moralwerttest zu bestehen, selbst wenn
  Slaaneshbestien den MW auf 0 gesenkt haben.

 \item
Moralwertabzüge durch den Verdammnisstein sind dauerhaft, selbst wenn das
betroffene Modell zwischenzeitlich stirbt und durch eine Sonderregel wieder ins
Spiel kommt. Wenn der Verdammnisstein den Moralwert eines Modells auf 0 senkt,
wird es aus dem Spiel entfernt. Dagegen hilft keine Sonderregel, das Modell ist
dauerhaft aus dem Spiel entfernt.

 \item Wenn ein Modell mit mehreren Lebenspunkten einen Profilwertest, ausgelöst
  durch Ekelerregende Heimsuchung oder Pavane des Slaanesh, nicht besteht und
  dann noch lebt, endet die Psionische Attacke sofort.

 \item Jede nicht verhinderte Verwundung durch die Sonderregel Geisträuber oder
  Überträger zwingt das betroffene Modell(solange es noch lebt) zu einem
  erneuten Initiative- /Widerstandstest.

 \item Wenn eine Einheit mit Ikone des Chaos aus der Reserve erscheint, können
  nachfolgende Einheiten mit der Sonderregel Schocktruppen, diese Ikone sofort
  nutzen (das heißt im selben Spielzug), um die Abweichung beim Schocken zu
  reduzieren.

 \item Wenn ein Modell, das über die erhabene Belohnung Warpband verfügt, sich
  auf einem lahmgelegten Streitwagenbefindet und getötet wird, entferne das
  komplette Modell inklusive Streitwagen und platziere es in der aktiven
  Reserve. Das Modell muss dann das Spielfeld mittels Schocken betreten.

 \item Eine Portalglyphe kann bereits in dem Spielzug, in dem sie platziert,
  neue Einheiten generieren.

 \item Eine Portalglyphe hat keine bestimmte Höhe und kann durch die kleine
  Explosionsschablone dargestellt werden.

 \item Wenn ein Modell mit mutierender Warpklinge durch
  Gedankenkontrollskarabäen übernommen wird und in diesem Zustand ein
  Charaktermodell oder eine monströse Kreatur aus der eigenen Armee tötet, wird
  die Warp Mutation Sonderregel nicht angewendet.

 \item Wenn ein Modell, das durch Gedankenkontrollskarabäen übernommen wurde,
  ein Modell aus der eigenen Armee angreift (inklusive sich selbst) und dieses
  Modell hat die Sonderregel Brennendes Blut, kommt diese Regel nicht zur
  Anwendung.

 \item Der Bonus von +1 auf Verletzungen ignorieren Würfe durch die Sonderregel
  Warpflammen ist kumulativ mit anderen Quellen von Warpflammen. Beachte das der
  Verletzungen ignorieren Wurf niemals besser als 2+ sein kann.

 \item Wenn die dämonischen Belohnungen ausgewürfelt werden, werden die
  warpgeschmiedete Rüstung und die undurchdringliche Haut nicht als identisch
  betrachtet, bezüglich der Regel, dass ein Modell jede Belohnung nur einmal
  haben kann.

 \item Ein Modell mit der mutierenden Warpklinge muss die Waffe im Nahkampf
  genutzt haben, damit die Sonderregel der Waffe zur Anwendung kommen kann.

 \item Ein Modell mit der Blutklinge muss die Waffe im Kampf nutzen, damit die
  Sonderregel der Waffe zur Anwendung kommen kann.

 \item Die Sonderregeln Warpverderben und Warpmutation können im selben
  Initiativeschritt gegen ein und dasselbe Modell wirken.

\end{itemize}

\subsection{Chaos Space Marines}

\begin{itemize}

 \item Ein Modell ohne die Sonderregel ``Champion des Chaos'' darf heroische
  Interventionen durchführen, um ein Model mit der Regel ``Champion des Chaos''
  in einer Herausforderung abzulösen. (Regelbuch S. 65)

 \item Zwei unabhängige Charaktermodelle mit verschiedenen Malen des Chaos
  können sich gleichzeitig derselben Einheit ohne irgendein Mal des Chaos
  anschließen.

 \item Psi-Waffen können Kharn den Verräter nicht sofort ausschalten, auch nicht
  wenn deren Stärke doppelt so hoch wie Kharn's Widerstand ist. (Seite 59, CSM
  Codex)

 \item Ein Kriegsschmied kann kein feindliches Fahrzeug verfluchen, während er
  in einem Transportfahrzeug ist, nicht mal wenn das Fahrzeug eine Feuerluke
  hat. (Seite 34, CSM Codex)

 \item Ein Modell, das einen durch den schwarzen Streitkolben verursachten
  Widerstandstest verpatzt, wird aus dem Spiel entfernt, selbst wenn es über die
  Sonderregel ``Ewiger Krieger'' verfügt. (Seite 69, CSM Codex)

 \item Wenn ein im Nahkampf gebundenes CSM Modell am Ende der Nahkampfphase,
  innerhalb von 3 Zoll keinen Gegner mehr hat, weil diese den durch den
  schwarzen Streitkolben verursachten Widerstandstest verpatzt haben, gilt das
  Modell dennoch solange als im Nahkampf gebunden, bis alle gegnerischen
  Modelle, die an diesem Nahkampf beteiligt sind, entfernt wurden. (Seite 69,
  Codex CSM)

 \item Wenn Typhus seinen ``Vernichterschwarm'' entfesselt, während er in einer
  Herausforderung kämpft, platziere die entsprechende Schablone ganz normal über
  Typhus. Das Modell, das mit Typhus in der Herausforderung kämpft, gilt als
  eigene Einheit, unabhängig von anderen Einheiten/Modellen die von der
  Schablone betroffen sind. Verwundungen werden dann normal zugeteilt, für jede
  von der Schablone betroffene Einheit. Verluste werden normal entfernt,
  beginnend mit den Modellen, die Typhus am nächsten sind. Der
  ``Vernichterschwarm'' kann so auch Modelle verwunden, die nicht in der
  Herausforderung kämpfen. (Seite 61, Codex CSM)

 \item Seuchenzombies können Waffenstellungen besetzen und deren Waffen
  abfeuern.  (Regelbuch S. 105)

 \item Die Sichtlinie für den Höllendrachen wird von seinem Kopf gezogen. Der
  Kopf hat einen 360 Grad Sichtbereich und ignoriert den Rest des Modells, wenn
  die Sichtlinie gezogen wird. (CSM FAQ)

 \item Der Dimensionschlüssel kann nicht den Effekt des Störsenders, der zu
  einem Land Speeder Storm gekauft wurde, außer Kraft setzen. (Seite 69, Codex
  CSM) 

\end{itemize}

\subsection{Dark Angels}

\begin{itemize}

 \item Eine Einheit, die Belial enthält, weicht nicht ab, wenn sie die
  Psi-Fähigkeit „Tor zur Unendlichkeit" benutzt. Gleichzeitig kann kein Modell
  der Einheit vom Warp verschlungen werden.

 \item Ezekiel erhält durch das Buch der Erlösung kein KG 6.

 \item Der Landspeeder Darkshroud verhüllt auch sich durch den Mantel der Engel
  selbst, d.h. er erhält Schleier und Tarnung.

 \item Der Widerstandmalus von -1 durch den Ravenwing-Granatwerfer wirkt sofort.
  Beispielsweise verwunden andere Schüsse der Einheit, die den Granatwerfer
  benutzt hat, besser.

\end{itemize}

\subsection{Eldar}

\begin{itemize}

 \item 
  \begin{minipage}[t]{0.65\textwidth}
Das Serpentschild ist ein Ausrüstungsgegenstand. Als Mündung zählt eine der
vorderen Spitzen des Serpents. Als Waffenlauf wird die Innenseite des Rumpfes
genommen. 
\end{minipage} \
  \begin{minipage}[t]{0.35\textwidth}
\includegraphics[height=2cm, valign=T, margin=0.0cm]{Bilder/serpent_transparent.png}
\end{minipage}

\end{itemize}

\subsection{Inquisition}

\begin{itemize}

 \item Schnellschüsse durch den Träger des Psyocculum (oder seiner Einheit) auf
  eine Einheit die mindestens ein Modell mit der Sonderregel Psioniker,
  Psionischer Pilot oder einer Bruderschaft aus Psioniker/Hexern werden mit BF 1
  durchgeführt.

\end{itemize}

\subsection{Space Marines}

\begin{itemize}

 \item Ein Kontingent der Black Templars benutzt den entsprechenden Eintrag aus
  der Alliiertenmatrix des Regelbuchs.

 \item Der W6 von Grav-Waffen gegen Fahrzeuge zählt in allen Belangen als
  Panzerungsdurchschlagswurf. Beispielsweise können Deckungswürfe durchgeführt
  werden.

 \item Grav-Waffen nutzen zum Verwunden den RW der Mehrheit der Einheit. Bei
  gleichstand wird der höhere (schlechtere) RW genommen.

 \item Grav-Waffen nutzen zum Verwunden den in der Phase aktuellen RW der
  Zieleinheit.

 \item Jede Ordensreliquie darf nur einmal pro Armee gewählt werden; unabhängig
  davon, wie viele Kontingente eingesetzt werden.

 \item Ein Modell darf nur mit einem Gegenstand von der Liste für
  Ordensreliquien ausgerüstet werden.

\end{itemize}

\subsection{Sternenreich der Tau}

\begin{itemize}

 \item Ein Schildgenerator kostet 25 Punkte.

 \item Der Wurf für Aun'Vas Paradoxon of Dualität kann zusätzlich zu jedem
  anderen Schutzwurf durchgeführt werden. Waffen mit der allgemeinen Sonderregel
  ``Deckung ignorieren'' negieren nicht den Effekt des Paradoxon der Dualität..

 \item Eine Einheit, die von der Inkarnation der Elemente eines Himmlischen
  profitieren möchte, muss sich in 12“ zu ihm befinden, wenn sie dies nutzen
  möchte.  Beispielsweise muss im Falle des Anmut des Zephyrs die Rennbewegung
  innerhalb von 12“ zum Himmlischen enden um die Schnellschüsse abgeben zu
  können.

 \item Ein Geistteam muss sechs Modelle (Drohnen ausgenommen) umfassen, um den
  zweiten Fusionsblaster kaufen zu können.

 \item Ein Sonnenhai-Bomber beginnt das Spiel mit einer Pulsbombengenerator.

 \item Wenn eine Abfangdrohne aus einem Sonnenhai-Bomber aussteigt, hat die
  Bewegungsdistanz des Bombers keine Auswirkung auf die Schussattacke der
  Drohne.

 \item Schüsse eines Modell mit der Sonderregel Flugabwehr, dass eine Lenkrakete
  mit Hilfe eines Zielmarkierer und der Sonderregel ``Ziel erfasst'' auf einen
  Flieger oder eine fliegende monströse Kreatur im Flugmodus abfeuert, werden
  mit BF5 abgehandelt.
Schüsse von Modellen, die kein Flugabwehr haben, werden mit BF1 abgehandelt.

 \item Wenn ein Modell mit der Sonderregel Flugabwehr auf Flieger oder monströse
  fliegende Kreaturen im Flugmodus schießt und Lenkraketen mithilfe der
  Sonderregel ``Zielführung'' abfeuert, werden diese Schüsse BF5 abgehandelt.  Hat
  das Modell nicht Sonderregel Flugabwehr nicht, werden die Schüsse mit BF1
  abgehandelt.

 \item Ein Fahrzeug, dass nur Schnellschüsse abgeben kann, kann auch Lenkraketen
  mithilfe der Sonderregel ``Ziel erfasst'' nur mit BF1 abfeuern.

 \item Wenn mehrere Einheiten Abwehrfeuer geben und Zielmarkierer platziert
  werden, können diese durch eine andere Einheit, die später schießt genutzt
  werden.

 \item Für ein Punktverteidigungs-Zielrelais zählen Zielmarkierer als Waffen mit
  S5 oder weniger.

 \item Sowohl der Drohnensteuerung als auch das Gegenfeuer-Abwehrsystem
  Unterstützungssystem haben keine Auswirkung auf Drohnen, die Schnellschüsse
  abgeben.

 \item Multiple Zielerfassung kann im Abwehrfeuer nicht dazu benutzt werden, um
  eine Einheit zu beschießen, die nicht angreift.

\end{itemize}

\subsection{Tyraniden}

\begin{itemize}

 \item Ein Morgon, der die Fähigkeit Schrecken aus der Tiefe benutzt,
\begin{itemize}
 \item Der Morgon kann auch so schocken, dass er Modelle trifft, die im Nahkampf
  gebunden sind
 \item Wenn der Morgon einen Transporter zerstört, steigen die Insassen normal
  aus, bevor die 2. Schablone bzw der Morgon platziert wird. Dies kann dazu
  führen, dass der Morgon ein Schocktruppenmissgeschick erleidet.
 \item Wenn der Morgon platziert wird, ist die Mitte seines Bases über der Mitte
  der Schablone. Die Blickrichtung ist frei wählbar. Ein kleiner Teil des Bases
  wird über die Schablone hinausragen.
\end{itemize}

 \item Die Flügel der Schwarmdrude werden bei Überflugattacken ignoriert. Nutze
  den Kopf, Körper und das Base des Models.

 \item Die Schablone Geiferkanone der Schwarmdrude wird wie bei Flammenwaffen
  üblich am Base angelegt.

 \item Sporenminen, die durch Sporensalve erschaffen wurden, können in dieser
  Runde angreifen. Beachte, dass Sporenminen nicht angreifen dürfen, wenn sie
  geschockt sind.

 \item Die Entflammbar Sonderregel der Pyrovore betrifft nur Modelle in
  W6“-Umkreis.

\end{itemize}

