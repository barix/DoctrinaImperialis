\section{Warhammer 40000 Regelbuch}

\subsection{Grundlegende Prinzipien}

\begin{itemize}

 \item Jeder sollte das erste Kapitel des Regelbuchs kennen. Besonderes
  Augenmerk sollte hierbei auf die ``allerwichtigste Regel'' (S. 4) und auf
  ``Kippe liegende Würfel'' (S. 10) gelegt werden.

\item Klärt vor jedem Spiel, wie ihr mit Kippe liegende Würfeln umgehen wollt.
 Seid ihr euch uneins, so sind alle Kippe liegenden Würfel ungültig und werden
 daher ignoriert und \textbf{immer} neu gewürfelt.

\end{itemize}

\subsection{Modifikatoren}

\begin{itemize}

 \item Wie auf Seite 8 des Regelbuchs beschrieben, gilt bei
  Profilwertmodifikationen Punktrechnung geht \textbf{vor} Strichrechnung.

\end{itemize}

\subsection{Der Geist des Spiels}

\begin{itemize}

 \item Es ist genau das: ein Spiel und ein Hobby, dass \textbf{Spaß} machen
  soll.  Bedenke das, wenn du spielst.

\end{itemize}

\subsection{Sichtlinien}

\begin{itemize}

 \item Allgemein sollte man beim Sichtlinien messen großzügig sein. Das
  beschleunigt das Spiel immens. Im Zweifelsfall kann das Modell beschossen
  werden, wie auf S. 15. Ein Deckungswurf wird gewährt.

\item Bei Fahrzeugen wird die Sichtlinie von der Waffenaufhängung gemessen. Bei
 allen anderen wie beschrieben von den Augen zum Körper des Ziels gemessen.

\end{itemize}

\subsection{Das wacklige Modell Syndrom}

\begin{itemize}

 \item Es wird genau so verfahren, wie auf S. 20 im Regelbuch beschrieben.
  Zusammengefasst müssen sich alle Spieler darüber im Klaren sein, wo ein Modell
  gerade steht.  Wenn das Modell ständig umfällt, wird ein Marker platziert, der
  die Kante des Bases/Modells widerspiegelt.

 \item Zum Messen von Distanzen und Sichtlinien wird wie im Kasten beschrieben
  das Modell an die Stelle gehalten.

\end{itemize}
