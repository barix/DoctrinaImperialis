\section{Warhammer 40000 Regelbuch}

\subsection{Grundlegende Prinzipien}

\begin{itemize}

 \item Jeder sollte das erste Kapitel des Regelbuchs kennen. Besonderes
  Augenmerk sollte hierbei auf die ``allerwichtigste Regel'' (S.4) und auf ``Kippe
  liegende Würfel'' (S.5) gelegt werden.

\item Klärt vor jedem Spiel, wie ihr mit Kippe liegende Würfeln umgehen wollt.
 Seid ihr euch uneins, so sind alle Kippe liegenden Würfel ungültig und werden
 daher ignoriert und \textbf{immer} neu gewürfelt.

\end{itemize}

\subsection{Modifikatoren}

\begin{itemize}

 \item Wie auf Seite 2 des RB beschrieben, gilt bei Profilwertmodifikationen
  Punktrechnung geht \textbf{vor} Strichrechnung.

\end{itemize}

\subsection{Der Geist des Spiels}

\begin{itemize}

 \item Es ist genau das: ein Spiel und ein Hobby, dass \textbf{Spaß} machen
  soll.  Bedenke das, wenn du spielst.

\end{itemize}

\subsection{Sichtlinien (S. 8)}

\begin{itemize}

 \item Allgemein sollte man beim Sichtlinien messen großzügig sein. Das
  beschleunigt das Spiel immens. Im Zweifelsfall kann das Modell beschossen
  werden, wie auf S.8 unten links beschrieben. Ein Deckungswurf wird gewährt.

\item Bei Fahrzeugen wird die Sichtlinie von der Waffenaufhängung gemessen. Bei
 allen anderen wie beschrieben von den Augen zum Körper des Ziels gemessen.

\end{itemize}

\subsection{Das wacklige Modell Syndrom}

\begin{itemize}

 \item Es wird genau so verfahren, wie auf S.11 im RB beschrieben.
  Zusammengefasst müssen sich alle Spieler darüber im Klaren sein, wo ein Modell
  gerade steht.  Wenn das Modell ständig umfällt, wird ein Marker platziert, der
  die Kante des Bases/Modells widerspiegelt.

 \item Zum Messen von Distanzen und Sichtlinien wird wie im Kasten beschrieben
  das Modell an die Stelle gehalten.

\end{itemize}

\subsection{Die Bewegungsphase}

\begin{itemize}

 \item Wenn sich ein Modell bewegt, wird eine gerade, dreidimensionale Linie
  zwischen der Startposition des Bases und der Endposition gezogen. Dies trifft
  für jede Bewegung zu, egal in welcher Phase die Bewegung stattfindet.

 \item Ein Panzer darf sich durch feindliche, nicht im Nahkampf gebundene
  Modelle bewegen, wenn er sich auf der Stelle dreht, um die Richtung für einen
  Panzerschock (S.85) zu bestimmen. Einheiten, durch die sich auf diese Weise
  bewegt wurde, zählen als seien sie von einem Panzerschock betroffen und dürfen
  den Regeln entsprechend eine ``Tod oder Ehre''-Attacke durchführen.
  (Vorausgesetzt die Einheit besteht ihren Moraltest)

\end{itemize}

\subsection{Die Schussphase}

\begin{itemize}

 \item Modelle, die die Turboboost-Fähigkeit einsetzen, können sich in der
  Schussphase eine beliebige Anzahl in Zoll (bis zum Maximum ihres
  Einheitentyps) und in eine beliebige Anzahl von Richtungen bewegen. Dies kann
  dazu führen, dass ihre Turboboost-Bewegung an exakt der selben Stelle endet,
  an der sie begann.

 \item Es können Modellen Wunden zugeteilt und diese als Verlust entfernt
  werden, sofern sie sich zu Beginn der Schussattacke in Reichweite und
  Sichtlinie zu mindestens einem schießenden Modell befanden, wenn die
  Trefferwürfe durchgeführt werden.
\begin{itemize}

 \item \textit{Beispiel 1:} Eine Einheit bestehend aus 10 Graumähnen mit 8 Bolter und 2
  Meltern feuert auf einen Trupp Chaos Space Marines. Der Space-Wolves-Spieler
  entscheidet sich die Bolterwunden zuerst abzuhandeln. Nachdem die
  Rüstungswürfe für die Bolterwunden abgehandelt wurden, ist der dichteste
  Chaosmarine außerhalb der 12“-Reichweite beider Melter. Der CSM können die
  Melterwunden zugeteilt werden und er kann als Verlust entfernt werden,
  solange irgendeiner der mit Bolter ausgerüsteten Graumähnen in Reichweite und
  eine Sichtlinie ziehen konnte, als die Trefferwürfe durchgeführt wurden.

 \item \textit{Beispiel 2:} Ein Taktischer Trupp der Space Marines schießt mit 4 Boltern
  und einer Laserkanone auf eine Einheit der Orks. Die 4 Bolter sind in
  Reichweite und haben Sichtlinie zu einem einzigen Ork. Die Laserkanone hat
  Reichweite und ist in Sichtlinie zu der gesamten Orkeinheit. Alle Orks der
  Einheit können nicht verhinderte LP-Verluste zugewiesen bekommen und als
  Verlust entfernt werden.

\end{itemize}

\item Beachte, dass nicht verhinderte LP-Verluste von Waffen, die keine
 Sichtlinie benötigen (z.B. Sperrfeuer oder Tau Schwärmer Raketensysteme), auch
 Modellen zugewiesen werden können, zu denen kein Modell der feuernden Einheit
 eine Sichtlinie ziehen kann.

\item Fahrzeuge, zu denen von einer schießenden Einheit keine Sichtlinien
 gezogen werden können, können nicht von den Schüssen getroffen werden, außer
 natürlich die Schüsse benötigen keine Sichtlinie

\item Außer es wird explizit erwähnt, dürfen Modelle, denen es gestattet ist in
 der \textbf{Schussphase} mehr als eine Waffe abzufeuern, in anderen Phasen des
 Spiels von dieser Regel keinen Gebrauch machen.

\end{itemize}

\subsection{Die Nahkampfphase}

\begin{itemize}

 \item Eine angreifende Einheit würfelt normal ihre 2W6-Angriffsbewegung aus,
  bevor bestimmt wird, ob irgendein angreifendes Modell sich durch schwieriges
  Gelände bewegen muss. Danach, vor irgendeiner Bewegung eines angreifenden
  Modells, müssen beide Spieler sich ausmalen, wo sich jeder Angreifer
  hinbewegen kann und wird. Wenn sich in diesem Prozess abzeichnet, dass der
  Angriff durch schwieriges Gelände erfolgt, wird der zusätzliche W6 geworfen,
  wie auf S. 22 beschrieben.  Wenn die angreifende Einheit eine Fähigkeit hat,
  die es ihr erlaubt einen oder mehrere Angriffswürfel neu zu werfen (wie z.B.
  Sprinten), kann diese Fähigkeit einsetzen entweder bevor oder nachdem der
  zusätzliche W6 für schwieriges Gelände geworfen wurde. Keinesfalls beides.
\begin{itemize}
 \item \textit{Beispiel:} Eine Einheit mit Sprinten würfelt 2W6 für ihre
  Angriffsreichweite und erwürfelt 3“. Dies reicht nicht, um das Ziel zu
  erreichen und der Spieler entscheidet sich beide Würfel neu zu werfen und
  erhält dieses Mal 11“. Unglücklicherweise sorgt die neue Angriffsreichweite
  dafür, dass der Angriff durch schwieriges Gelände erfolgt, sodass nun der
  zusätzliche W6 für einen Angriff durch schwieriges Gelände geworfen werden
  muss. Das Ergebnis dieses Wurfs ist eine 1. Die Einheit kann dieses W6 nicht
  wiederholen, da es die Wiederholung durch Sprinten bereits eingesetzt hat.
\end{itemize}

\item Sobald ein Modell einer angreifenden Einheit einen Test für gefährliches
 Gelände ablegen muss, zählt der Angriff als ein Angriff durch schwieriges
 Gelände, wie auf S. 22 RB beschrieben. Beispielsweise erleiden Sprungtruppen,
 die ohne Offensivgranaten angreifen den Initiative-Malus, sobald ein Modell die
 Angriffsbewegung in schwierigen Gelände startet oder beendet.

\item Wenn auf eine angreifende Einheit mehrere Einheiten Abwehrfeuer geben
 dürfen, so handelt der feuernde Spieler Abwehrfeuer für eine Einheit seiner
 Wahl ab, bevor er sich entscheidet, ob er mit einer weiteren Einheit
 Abwehrfeuer ab handelt.  Fahrt solange fort, bis jede erlaubte Einheit entweder
 Abwehrfeuer geschossen hat, oder sich der Spieler dagegen entschieden hat.

\item Einheiten zählen als im Nahkampf gebunden, sobald ein gegnerisches Modell
 sich in Basekontakt befindet. Dies hindert die Einheit daran Abwehrfeuer gegen
 eine Einheit zu geben, die später in dieser Phase angreift.

\item Nach einem gewonnen Nahkampf darf sich eine Einheit entscheiden, sich
 nicht neu zu positionieren. Wenn dies entschieden wird, verharren alle Modelle
 der Einheit an ihrer Position.

\item Wenn sich eine Einheit dazu entschließt, eine Neupositionierungs-Bewegung
 durchzuführen, muss sie am Ende dieser Bewegung zu allen feindlichen Modellen
 1“-Abstand halten. Dies schließt auch Fahrzeuge ein, die eventuell diese Phase
 attackiert wurden.

\item Die Nachrückbewegungen am Ende des Nahkampfes werden immer durchgeführt.
 Man muss also alle Modelle des Trupps 3“ dichter an den Feind bewegen. Es
 beginnt hierbei der Spieler, dessen Spielerzug gerade läuft.  Sollte diese
 Bewegung nicht ausreichen, um in Basekontakt zu kommen, kann wie auf S.27
 beschrieben eine Neupositionierung-Bewegung durchgeführt werden.

\end{itemize}

\subsection{Charaktere}

\begin{itemize}

 \item Wunden durch präzise Schüsse/Hiebe bilden immer eine eigene (Wund-)Gruppe
  in einem Wundpool

 \item Unabhängige Charaktermodelle können Ein-Modell-Einheiten nur dann
  beitreten, wenn diese Einheit in ihrem Armeelisteneintrag die Option auf
  mehrere Modelle hat (z.B. ein UCM kann einem XV104 Sturmflut beitreten,
  selbst wenn dieser ohne Drohnen das Spiel als Einzelmodell begann).

\end{itemize}

\subsection{Psioniker/Psionische Fähigkeiten}

\begin{itemize}

 \item ``Gefahren des Warp''-Treffer können nicht mit ``Achtung, Sir'' weiter
  gegeben werden.

 \item Wenn die Psi-Fähigkeit ``Entsetzen hervorrufen'' eine im Nahkampf gebundene
  Einheit zum Rückzug zwingt, verlässt diese den Nahkampf und macht eine normale
  Fluchtbewegung. Falls diese Einheit die einzige war, die den Gegner im Nahkampf
  band, so darf dieser weder nachsetzen noch sich neupositionieren.

 \item Mehrere gleiche Flüche, die von unterschiedlichen Modellen gesprochen
  wurden, sind kombinierbar.

 \item Ein Flieger im Normalschub erleidet nicht den Stärke-1-Treffer mit der
  Impuls-Sonderregel der Psi-Fähigkeit ``Maledictum Mechanicum''

 \item Ein Psioniker, der sich entweder zu Beginn seines Zuges auf der Flucht
  befindet oder sich in seiner Bewegungsphase nicht bewegen kann, darf die
  Psi-Fähigkeit ``Tor zur Unendlichkeit'' diesen Zug nicht benutzen.

\end{itemize}

\subsection{Allgemeine Sonderregeln}

\begin{itemize}

 \item Modelle, die bei einer Zurückfallen-Bewegung auf ein Hindernis treffen,
  beenden sofort ihre Bewegung, wenn sie es nicht über- oder durchqueren können,
  wie z.B.  unpassierbares Gelände oder Tischkanten.

 \item UCM mit der Sonderregel ``Infiltrieren'' dürfen sich einer Einheit ohne
  diese Sonderregel anschließen, was der Einheit erlaubt, zusammen mit dem UCM
  zu infiltrieren.

 \item Ein Modell, dass die Abfangen-Sonderregel benutzt, muss:
\begin{itemize}
 \item eine Sichtlinie zum Ziel haben, selbst wenn die Waffe diese normalerweise
  nicht benötigt
 \item zählt für diese Schussattacke immer als stationär
 \item darf im selben Spielerzug mit der selben Waffen immer noch Abwehrfeuer
  geben
\end{itemize}

\item Eine Einheit kann eine Überflugattacke durchführen, wenn sie den Luftraum
 verlässt.

\item Attacken eines Modells mit der Sonderregel ``Wuchtige Hiebe'', werden mit DS
 2 abgehandelt. Selbst wenn es eine Nahkampfwaffe benutzt, die einen Durchschlag
 von 3 oder schlechter hat.

\item Attacken, die bei einem Verwundungswurf von 6 sofort ausschalten, müssen
 das Ziel verwunden können, damit die Sonderregel ``Sofort ausschalten'' zum
 Tragen kommt, z.B. eine S4-Attacke kann ein W8-Modell nicht verwunden.

\item Eine Einheit, die das Ergebnis ``Am falschen Ort'' von der
 Schocktruppen-Missgeschicktabelle erleidet, muss vom gegnerische Spieler so
 platziert werden, dass sie kein Schocktruppenmissgeschick erleidet. Sollte dies
 nicht möglich sein, so geht die Einheit in aktive Reserve.

\item Eine Einheit mit der Sonderregel ``Eine Bruderschaft von
 Psionikern/Hexern'', bestimmt ihre Reichweite und Sichtlinie von einem Modell
 mit der Sonderregel, nach Wahl des Spielers. Wenn ein Hexenfeuer gewirkt wird,
 zählt nur das gewählte Modell als hätte es geschossen, d.h. alle anderen
 Modelle der Einheit können ihre Schusswaffen abfeuern.

\item Es dürfen keine Deckungswürfe gegen Streif- oder Volltreffer durchgeführt
 werden, die die Sonderregel ``Deckung ignorieren'' haben

\item Eine Geschützwaffe mit der Sonderregel ``Panzerjäger'' muss beide
 Panzerungsdurchlagswürfe wiederholen oder keinen.

\end{itemize}

\subsection{Flieger und fliegende monströse Kreaturen}

\begin{itemize}

 \item Attacken und Spezialfähigkeiten, die Schaden verursachen können, ohne
  einen Trefferwurf durchzuführen, haben keinen Effekt auf Flieger oder
  fliegende, monströse Kreaturen im Flugmodus.
Dies trifft auch dann zu, wenn die attackierende Einheit die Sonderregel
``Flugabwehr'' besitzt. Beachte, dass die Fähigkeit ``Sturmbringer'' von Imothek dem
Sturmherren eine explizite Ausnahme zu dieser Regelung ist.

\end{itemize}

\subsection{Explosivwaffen}

\begin{itemize}

 \item Fahrzeuge werden von Explosivwaffen getroffen, auch wenn die feuernde
  Einheit keine Sichtlinie zum Fahrzeug ziehen kann.

\end{itemize}

\subsection{Flammenwaffen}

\begin{itemize}

 \item Modelle, zu denen keine Sichtlinie gezogen werden kann, können von
  Flammenwaffen getroffen werden und somit Wunden zum Wundpool beisteuern. Wenn
  die Flammenwaffe eine Sichtlinie benötigt, können Modellen, die komplett außer
  Sicht sind, keine Wundverluste zugeteilt werden.
Gleichbedeutend können Fahrzeuge, die komplett außer Sicht zu einer Einheit ist,
die eine Flammenwaffe benutzt, nicht getroffen werden.

\item Bei Flammenwaffen mit der ``Schwall''-Sonderregel gelten alle Restriktionen
 zum Abfeuern von Flammenwaffen, auch wenn die die Flammenschablone nicht in
 Basekontakt mit dem feuernden Modell gelegt werden muss.

\end{itemize}

\subsection{Verbündete}

\begin{itemize}

 \item Eine Einheit, die ein unabhängiges Charaktermodell aus dem
  Verbündetenkontingent enthält, kann weder in ein Transportfahrzeug einsteigen
  noch das Spiel in einem beginnen.

 \item Reservewurfmodifikatoren betreffen Würfe für Einheiten aus demselben
  Kontingent, auch wenn diese ein unabhängiges Charaktermodell aus einem
  Verbündetenkontingent enthält.
Die Modifikatoren betreffen keine unabhängigen Charaktermodelle aus demselben
Kontingent, die an verbündete Einheiten angeschlossen sind.

\end{itemize}

\subsection{Befestigungen}

\begin{itemize}

 \item Eine Befestigung wird als Einheit der Armee des kontrollierenden Spielers
  behandelt. Daher wirken alle Sonderregeln und Fähigkeiten, die befreundete
  Einheiten/Modelle betreffen auch auf Befestigungen.
Sonderregeln und Fähigkeiten, die nur auf Einheiten/Modelle des eigenen Codex
wirken, betreffen nur Befestigungen, die aus jenem Codex stammen.
\begin{itemize}
 \item \textit{Beispiel 1:} Eine Befestigung in 6“-Umkreis zu einem Ork
  Späzialkraftfeld würde einen 5+ Deckungswurf erhalten (vorausgesetzt der
  Ork-Spieler hat sie gekauft)
 \item \textit{Beispiel 2:} Eine Befestigung in 3“-Umkreis zu einem Dark Angels
  Kraftfeldgenerator würde einen 4+ Rettungswurf erhalten.
\end{itemize}

\item Waffenstellungen sind keine Modelle oder Einheiten. Sie sind Gelände, was
 u. a.  heißt:
\begin{itemize}
 \item Sie generieren keine Siegpunkte, wenn sie zerstört werden.
 \item Sie profitieren nicht von Sonderregeln oder Fähigkeiten, die auf
  Einheiten/Modelle wirken (z.B. Dark Angels Kraftfeldgenerator)
 \item Sie werden nicht von Panzerschocks und Psi-Fähigkeiten betroffen
  (Ausnahme: Hexenfeuer)
\begin{itemize}
 \item Befestigungen, die Gebäude sind, kriegen einen Deckungswurf durch Gelände
  oder Einheiten, exakt wie bei Fahrzeugen
Waffenstellungen erhalten Deckungswürfe durch Gelände oder Einheiten, exakt wie
ein Infantriemodell
\item Einheiten können keine Angriffe auf Waffestellung ansagen. Jedoch kann ein
 Modell, dass eine andere Einheit angegriffen hat und in Basekontakt mit einer
 Waffenstellung ist, seine Attacken gegen diese richten, als wäre die
 Waffenstellung eine eigene Einheit.
\item Es können zwei Spieler Modelle in Basekontakt zu einer Waffenstellung
 haben (1“-Abstand beachten). Die Waffenstellung kann von beiden Modellen in
 aufeinander folgenden Schussphasen benutzt werden.
\item Nur Modelle die tatsächlich auf einen Himmelsschild-Landeplattform stehen,
 erhalten den 4+ Rettungswurf
\item Die Wände der Himmelsschild-Landeplattform gewähren einen 4+ Deckungswurf
\item Die Beine/Säulen der Himmelsschild-Landeplattform gewähren eine 3+
 Deckungswurf
 \item Die Regel ``Automatisches Feuer'' verlangt von eingebauten Waffen auf das
  nächste legale Ziel zu schießen, d. h. die Waffe muss  auf Ziele schießen, die
  es nicht verwunden/beschädigen kann. Sie darf nicht auf Ziele schießen, wenn
  andere Zielrestriktionen verletzt werden (z. B. Explosivschablonen über
  befreundeten Modellen platzieren)
\end{itemize}
\end{itemize}
\end{itemize}

\newpage

\subsection{Gelände}

\begin{itemize}

 \item Der 4+ Deckungswurf von Ruinen wird nur gewährt, wenn das Modell
  tatsächlich von den Mauern der Ruine verdeckt wird. Die Böden und das Base
  (sofern eines vorhanden ist) zählen als Geländezone.
Regeln, die explizit den Deckungswurf von Ruinen modifizieren, wirken nur auf
Deckungswürfe, die durch die Mauern der Ruine gewährt werden.

\item Wälder gewähren einen Deckungswurf, sobald eine Sichtlinie über
 irgendeinen Teil der Wald-Geländezone gezogen wird, auch wenn das Ziel nicht
 von irgendwelchen Bäumen/Sträuchern verdeckt ist.
Dies trifft auch auf feuernde Modelle zu, die mit einem Millimeter in der
Geländezone stehen und auf ein Ziel außerhalb schießen.

\end{itemize}

\subsection{Missionen}

\begin{itemize}

 \item Punktende Einheit sind immer auch verweigernde Einheiten

 \item Einheiten mit der Sonderregel ``Schwarm'' oder irgendeine andere Regeln,
  die einer Einheit  verbietet als punktend/verweigernd zu gelten, können in der
  Mission ``Große Kanonen ruhen nie'' nicht als punkten/verweigernd zählen.

 \item Wenn ein Relikt fallen gelassen wird, weil sich der Träger weiter als 6“
  bewegt hat, wird das Relikt an der Stelle platziert, an der der Träger seine
  Bewegung begann.

 \item Einheiten in Transportfahrzeugen können keinen Durchbruch erzielen

 \item Flieger können nicht Missionsmarker halten oder umkämpfen. Flieger, die
  zu punktenden Einheiten wurden, können dennoch Durchbruch erzielen.

\end{itemize}

\subsection{Landungskapseln}

\begin{itemize}

 \item Modelle, die aus einer Landungskapsel aussteigen, führen eine normale
  Bewegung durch, wobei sie am Ende dieser komplett in 6“ zur Kapsel befinden
  müssen

 \item Eine Kapsel verliert nicht automatisch einen Hüllenpunkt, wenn sie
  platziert wird. Wenn sie in schwierigem/gefährlichem Gelände landet und den
  Test für schwieriges Gelände verpatzt, verliert sie insgesamt 2 Hüllenpunkte.

 \item Im Spiel werden die Türen einer Kapsel komplett ignoriert. Sie blockieren
  keine Sichtlinien, werden nicht zum Messen beim Aussteigen genommen und
  feindliche Modelle müssen auch Abstand von 1“ zu ihnen einhalten.

 \item Einheiten, die eine Landungskapsel eingestiegen sind und denen sich ein
  unabhängiges Charaktermodell angeschlossen hat, werden ignoriert, wenn es
  darum geht, wie viele Einheiten in Reserve gehalten werden können
Das bedeutet, dass eine Landungskapselarmee zu Beginn des Spiels jede Einheit in
Reserve haben kann.

\end{itemize}

